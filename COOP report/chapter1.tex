\chapter{บทนำ}
\label{chapter:introduction}

\section{ที่มาและความสำคัญ}

บริษัท เน็กซี่ เทคโนโลยี จำกัด เป็นบริษัท รับจ้างพัฒนาซอฟต์แวร์ (Software House) ที่มีความต้องพัฒนา Digital Transformation โดยบริษัทใช้ Slack 
ในการติดต่อส่งข้อมูลกันภายในบริษัท Slack เป็นแอปพลิเคชันที่ใช้ในการติดต่อสื่อสารภายในองค์กร ที่มีส่วนเสริมที่ทำให้การทำงานในองค์กรมีความสะดวกยิ่งขึ้นอาธิ Slackbot 
ที่สามารถตอบคำตอบที่ตั้งไว้ ตามคำถามที่สามารถตรวจจับได้ โดยภายในบริษัทมีการแยก Channel ตามหัวข้อ เช่น \#calendar \#daily\_meeting เป็นต้น ห้อง \#calendar จะใช้ในการเขียนวันลา ขอเข้างานสาย หรือขอออกก่อนเวลา แต่ก็จะมีปัญหา เมื่อมีพนักงาน ลาล่วงหน้าหลายวัน ทำให้การเลื่อนหาเป็นไปได้ยาก ประกอบด้วย เว็บไซต์ภายในที่ดูแล Human resource ไม่มีตอบการใช้งานมากนัก เช่น หากมีพนักงานลาพักร้อน HR จำเป็นต้องเข้าเว็บไซต์เพื่อหักลบวันของพนักงานคนนั้นเอง ซึ่งมีโอกาสที่ข้อมูลถูกแก้ไขมีโอกาสผิดพลาดได้
\vskip1em
ทางบริษัท จึงสร้าง slackbot ที่มีความสามารถ จัดเก็บข้อมูลวันลาใน Database, นำไปหักลบจากจำนวนวันลาของพนักงานนั้น และแสดงผลพนักงานที่ลาวันนี้ทุก 10.00น ในวันทำงานเมื่อทำถึงจุดที่พนักงานทุกคนสามารถใช้งานแล้ว ภายในบริษัทมองเห็นว่ายังสามารถต่อยอดทำ Feature อื่นในการบริหารองค์กรได้อีก จึงได้ทำการเก็บ requirement จาก HR และสร้างเป็นโปรเจค Awesome HRM และเปลี่ยนชื่อเป็น ManyFox ในที่สุด
\vskip1em
ในรายงานฉบับนี้บริษัทต้องการที่จะพัฒนาระบบบริหารทรัพยากรบุคคลในองค์กรขนาดเล็ก ผ่าน Slack และควบคุมผ่าน เว็บไซต์ ภายใต้ชื่อแบรน ManyFox ที่ช่วยบริหารองค์กรด้าน การจัดการวันลากิจ ลาป่วย ลาพักร้อน และการเพิ่มลดจำนวนวันลาอัตโนมัติ และด้านการวางแผนงานในแต่ละวัน โดยพนักงานต้องเขียน สิ่งทำไปแล้ว
และสิ่งที่ทำในวันนี้ โดยจัดเก็บข้อมูลในรูปแบบเดียวกัน นอกจากสามารถใช้งานภายในบริษัทแล้ว บริษัทมีความต้องการขยายบริการให้สามารถใช้งานได้ในบริษัทลูกและองค์กรอื่นเช่นกัน

\section{วัตถุประสงค์ของการปฎิบัติงาน}
บริษัท Nextzy Technologies จำกัด มีความต้องการต่อยอด Slackbot ให้มีความสามารถจัดการบริหารองค์กรในนามโปรเจค ManyFoxที่ความสามารถกำหนดวันหยุดของบริษัท สามารถจัดการวันลาของพนักงาน และเขียนคำร้องวันลาส่งให้แก่หัวหน้าอัตโนมัติ สามารถประมวลผลข้อมูลการลาของบุคคลและของพนักงานทั้งบริษัท ในรูปแบบ Graph และสามารถ Export ข้อมูล และมีความต้องการทำให้สามารถรองรับหลายบริษัท

\section{ขอบเขตของการปฎิบัติงาน}
\begin{enumerate}
  \item พัฒนาระบบบบริหารทรัพยากรบุคคลขนาดเล็ก
  \item พัฒนา และแก้ไขปัญหาระบบอื่นที่บริษัทมอบหมายให้รับผิดชอบ
  \item ศึกษาวิธีและเทคนิคที่มีประสิทธิภาพเพื่อประยุกต์ใช้งาน
\end{enumerate}

\section{วิธีการดำเนินการวิจัย}

ทางบริษัทใช้ Agile Methodology ในดำเนินงาน โดยแบ่งออกเป็น Sprint  โดยแต่ละ Sprint จะมีระยะเวลา2-3อาทิตย์ และจะมีประชุมสรุปเมื่อจบ Sprint และรับหน้าที่ใน Sprint ใหม่ โดย Project manager จะเก็บรวบรวม Requirement และนำไปสร้าง User Story ( ความต้องการของ user ) และแจกแจงให้แก่คนในทีม เพื่อนำไปสร้าง task ของแต่ละคน โดยใช้ Taiga (Open source Project management) ในการจัดการ task และ เขียน Issues และกำหนดว่าให้ Developer คนใดแก้ไข



\section{ประโยชน์ที่คาดว่าจะได้รับ}
\begin{enumerate}
  \item นักศึกษาได้รับความรู้และเข้าใจในเรื่อง Web Development ทั้ง Frontend และ Backend ตาม User story และ Design ที่ได้รับ
  \item ระบบจัดการบริหารทรัพยากรบุคคล ManyFox ได้ถูกนำไปใช้ในหลายบริษัทชั้นนำในประเทศไทย
\end{enumerate}