\documentclass{itkmitlcoop}

\usepackage{afterpage}
\usepackage{graphicx,amsmath,latexsym,amssymb,amsthm}
\usepackage{indentfirst}
\usepackage{cite}
\usepackage{float}
\graphicspath{ {images/} }

\makeatletter
% \patchcmd{<cmd>}{<search>}{<replace>}{<succes>}{<failure>}
\patchcmd{\@chapter}{\addtocontents{lof}{\protect\addvspace{10\p@}}}{}{}{}% LoF
\patchcmd{\@chapter}{\addtocontents{lot}{\protect\addvspace{10\p@}}}{}{}{}% LoT
\makeatother

%1. Your thesis title (THAI)
\newcommand{\ThesisTiTle}{ระบบบริหารทรัพยากรบุคคล : ManyFox}
%2. Your thesis title (ENG)
\newcommand{\ThesisTiTleENG}{Awesome Human Resources Management : ManyFox}
%3. Your name
% Your prefix
\newcommand{\Gender}{นาย}
\newcommand{\AuName}{จักรภัทร บุญโรจน์}
%4. Your name ENG
\newcommand{\AuNameENG}{Jakkapat Boonroj}
%5. Your student Department 
\newcommand{\Depart}{วิศวกรรมสารสนเทศ}
%5. Your student Department eng
\newcommand{\DepartENG}{Information Engineering}
%5. Your student ID
\newcommand{\SId}{รหัสประจำตัว}
%6. Your advisor
\newcommand{\Advisor}{ผศ.มยุรี เลิศเวชกุล}
%6. Your advisor eng
\newcommand{\AdvisorENG}{Asst.Prof. Mayuree Lertwatechakul}
%6. Your Mentor
\newcommand{\Mentor}{ผู้นิเทศงาน}
%6. Your Mentor eng
\newcommand{\MentorENG}{Tanapruk Tangphianphan}
%7. Your advisor employee
\newcommand{\Exami}{ชื่อพนักงานที่ปรึกษา}
%8. สถานประกอบการ
\newcommand{\Company}{เน็กซี่เทคโนโลยี จำกัด}
%8. สถานประกอบการ
\newcommand{\CompanyENG}{Nextzy Technologies Co., Ltd}
%9. ภาคเรียนที่ (in normal letters)
\newcommand{\Sem}{1}
%10. ปีการศึกษา (in normal letters)
\newcommand{\AcaY}{2562}
%11. ปีการศึกษา (in normal letters)
\newcommand{\AcaYAD}{2019}
%12. วันส่งรายงาน
\newcommand{\SubD}{10 พฤศจิกายน พ.ศ. 2562}
%13. วันเริ่มทำงาน
\newcommand{\StartDWork}{5 สิงหาคม พ.ศ. 2562}
%14. วันสุดท้ายของการทำงาน
\newcommand{\EndDWork}{28 พฤศจิกายน พ.ศ. 2562}
%15. ที่อยู่สถานประกอบการ
\newcommand{\Address}{ที่อยู่}
%16. เว็บไซต์สถานประกอบการ
\newcommand{\Website}{เว็บไซต์}
%17. ตำแหน่งานที่ปฏิบัติ
\newcommand{\Position}{ตำแหน่งงานที่ปฏิบัติ}

\begin{document}
\frontmatter
\lhead{}\rhead{}\chead{}\lfoot{}\cfoot{\thepage}\rfoot{}
\makecover
\makethabstract{
    บริษัท \Company เป็นบริษัท Software House ที่มีความต้องพัฒนา Digital Transformation สำหรับเว็บไซต์และแอพพิเคชั่น ในระบบปฎิบัติการ Android และ iOS 
    โดยบริษัทต้องการพัฒนาระบบบริหารทรัพยากรบุคคลผ่าน Slack และ Website สำหรับใช้บริหารภายในองค์กร และเปิดให้เช่าบริการสำหรับองค์กรอื่น \\
    \textbf{ManyFox} เป็น Web application ที่เชื่อมต่อกับ Slack สำหรับใช้ในการจัดการวันหยุดและการลางานของพนักงาน การเขียนบันทึกประจำวัน (Daily meeting) 
    และ การจัดการข้อมูลของพนักงานภายในองค์กร โดยมีความสามารถเก็บข้อมูลและ นำเสนอข้อมลต่างๆผ่านทาง Slack รวมไปถึงการตั้งค่าของ บอทใน Slack ผ่านทาง Website
}
\makeengabstract{
    
}
\makeack{
    \begin{enumerate}
        \item นายสมเกียรติ กิจวงศ์วัฒนะ \quad ตำแหน่ง ผู้ประสานงานสหกิจศึกษา
        \item พี่ทรัส \quad ตำแหน่ง Project Manager
        \item นิชา \quad ตำแหน่ง Backend Developer Intern
        \item พี่ออน \quad ตำแหน่ง UX/UI Designer
        \item พี่บีม \quad ตำแหน่ง UX/UI Designer
        \item พี่ตุ้ย \quad ตำแหน่ง Backend Developer

    \end{enumerate}
}

% Setting margin for page numbering on frontmatter
\newgeometry{top=1in, bottom=1in, left=1.5in, right=1in, includefoot}

\newpage
\addcontentsline{toc}{chapter}{สารบัญ}
\tableofcontents

\newpage
\addcontentsline{toc}{chapter}{สารบัญตาราง}
\listoftables

\newpage
\addcontentsline{toc}{chapter}{สารบัญภาพ}
\listoffigures

% Reset frontmatter page numbering margin, back to original margin from class file
\restoregeometry

\mainmatter
\lhead{}\rhead{\thepage}\chead{}\lfoot{}\cfoot{}\rfoot{}

\chapter{บทนำ}
\label{chapter:introduction}

\section{ที่มาและความสำคัญ}

บริษัท เน็กซี่ เทคโนโลยี จำกัด เป็นบริษัท รับจ้างพัฒนาซอฟต์แวร์ (Software House) ที่มีความต้องพัฒนา Digital Transformation โดยบริษัทใช้ Slack 
ในการติดต่อส่งข้อมูลกันภายในบริษัท Slack เป็นแอปพลิเคชันที่ใช้ในการติดต่อสื่อสารภายในองค์กร ที่มีส่วนเสริมที่ทำให้การทำงานในองค์กรมีความสะดวกยิ่งขึ้นอาธิ Slackbot 
ที่สามารถตอบคำตอบที่ตั้งไว้ ตามคำถามที่สามารถตรวจจับได้ โดยภายในบริษัทมีการแยก Channel ตามหัวข้อ เช่น \#calendar \#daily\_meeting เป็นต้น ห้อง \#calendar จะใช้ในการเขียนวันลา ขอเข้างานสาย หรือขอออกก่อนเวลา แต่ก็จะมีปัญหา เมื่อมีพนักงาน ลาล่วงหน้าหลายวัน ทำให้การเลื่อนหาเป็นไปได้ยาก ประกอบด้วย เว็บไซต์ภายในที่ดูแล Human resource ไม่มีตอบการใช้งานมากนัก เช่น หากมีพนักงานลาพักร้อน HR จำเป็นต้องเข้าเว็บไซต์เพื่อหักลบวันของพนักงานคนนั้นเอง ซึ่งมีโอกาสที่ข้อมูลถูกแก้ไขมีโอกาสผิดพลาดได้
\vskip1em
ทางบริษัท จึงสร้าง slackbot ที่มีความสามารถ จัดเก็บข้อมูลวันลาใน Database, นำไปหักลบจากจำนวนวันลาของพนักงานนั้น และแสดงผลพนักงานที่ลาวันนี้ทุก 10.00น ในวันทำงานเมื่อทำถึงจุดที่พนักงานทุกคนสามารถใช้งานแล้ว ภายในบริษัทมองเห็นว่ายังสามารถต่อยอดทำ Feature อื่นในการบริหารองค์กรได้อีก จึงได้ทำการเก็บ requirement จาก HR และสร้างเป็นโปรเจค Awesome HRM และเปลี่ยนชื่อเป็น ManyFox ในที่สุด
\vskip1em
ในรายงานฉบับนี้บริษัทต้องการที่จะพัฒนาระบบบริหารทรัพยากรบุคคลในองค์กรขนาดเล็ก ผ่าน Slack และควบคุมผ่าน เว็บไซต์ ภายใต้ชื่อแบรน ManyFox ที่ช่วยบริหารองค์กรด้าน การจัดการวันลากิจ ลาป่วย ลาพักร้อน และการเพิ่มลดจำนวนวันลาอัตโนมัติ และด้านการวางแผนงานในแต่ละวัน โดยพนักงานต้องเขียน สิ่งทำไปแล้ว
และสิ่งที่ทำในวันนี้ โดยจัดเก็บข้อมูลในรูปแบบเดียวกัน นอกจากสามารถใช้งานภายในบริษัทแล้ว บริษัทมีความต้องการขยายบริการให้สามารถใช้งานได้ในบริษัทลูกและองค์กรอื่นเช่นกัน

\section{วัตถุประสงค์ของการปฎิบัติงาน}
บริษัท Nextzy Technologies จำกัด มีความต้องการต่อยอด Slackbot ให้มีความสามารถจัดการบริหารองค์กรในนามโปรเจค ManyFoxที่ความสามารถกำหนดวันหยุดของบริษัท สามารถจัดการวันลาของพนักงาน และเขียนคำร้องวันลาส่งให้แก่หัวหน้าอัตโนมัติ สามารถประมวลผลข้อมูลการลาของบุคคลและของพนักงานทั้งบริษัท ในรูปแบบ Graph และสามารถ Export ข้อมูล และมีความต้องการทำให้สามารถรองรับหลายบริษัท

\section{ขอบเขตของการปฎิบัติงาน}
\begin{enumerate}
  \item พัฒนาระบบบบริหารทรัพยากรบุคคลขนาดเล็ก
  \item พัฒนา และแก้ไขปัญหาระบบอื่นที่บริษัทมอบหมายให้รับผิดชอบ
  \item ศึกษาวิธีและเทคนิคที่มีประสิทธิภาพเพื่อประยุกต์ใช้งาน
\end{enumerate}

\section{วิธีการดำเนินการวิจัย}

ทางบริษัทใช้ Agile Methodology ในดำเนินงาน โดยแบ่งออกเป็น Sprint  โดยแต่ละ Sprint จะมีระยะเวลา2-3อาทิตย์ และจะมีประชุมสรุปเมื่อจบ Sprint และรับหน้าที่ใน Sprint ใหม่ โดย Project manager จะเก็บรวบรวม Requirement และนำไปสร้าง User Story ( ความต้องการของ user ) และแจกแจงให้แก่คนในทีม เพื่อนำไปสร้าง task ของแต่ละคน โดยใช้ Taiga (Open source Project management) ในการจัดการ task และ เขียน Issues และกำหนดว่าให้ Developer คนใดแก้ไข



\section{ประโยชน์ที่คาดว่าจะได้รับ}
\begin{enumerate}
  \item นักศึกษาได้รับความรู้และเข้าใจในเรื่อง Web Development ทั้ง Frontend และ Backend ตาม User story และ Design ที่ได้รับ
  \item ระบบจัดการบริหารทรัพยากรบุคคล ManyFox ได้ถูกนำไปใช้ในหลายบริษัทชั้นนำในประเทศไทย
\end{enumerate}
\chapter{เอกสารและงานวิจัยที่เกี่ยวข้อง}
\label{chapter:related-theory}

\section{ทฤษฎีและเนื้อหาที่เกี่ยวข้อง}
\subsection{การบริหารจัดการทรัพยากรมนุษย์}
การบริหารจัดการทรัพยากรมนุษย์ (Human Resources Management) คือกระบวนการที่จัดการนำเป้าหมายของบุคลากรและเป้าหมายขององค์กรให้มาบรรจบกัน เพื่อผลสำเร็จร่วมกันของทั้งองค์กรและบุคคล โดยมุ่งเน้นไปที่ผลของการจัดการ ส่งเสริมและช่วยพัฒนาศักยภาพของพนักงงานอย่างเต็มความสามารถ
\subsection{กฎหมายคุ้มครองแรงงาน}
ตามกฎหมายคุ้มครองแรงงาน วันหยุดพักผ่อนประจำปี ต้องมีไม่ต่ำกว่า 6 วันต่อปีสำหรับลูกจ้างที่ทำงานติดต่อกันมาครบ 1 ปี และลูกจ้างสามารถ ลาป่วย ลากิจ ลาทำหมัน ลารับราชการทหาร ลาคลอดบุตร และลาฝึกอบรมได้

\section{เทคโนโลยีและภาษาที่ใช้ในการพัฒนา}
\subsection{HTML}
HTML ย่อมาจาก Hypertext Markup Language  คือภาษาสากลในการสร้างหน้า Web page
\begin{figure}[!h]
	\centering
	\includegraphics[width=0.35\linewidth]{html}
	\caption{Hypertext Markup Language}
	\label{Fig:HTML}
\end{figure}
\newpage
\subsection{CSS}
CSS ย่อมาจาก Cascading Style Sheets เป็นภาษาจัดการหน้าตาให้กับเอกสาร ให้กับ Markup Language เช่่น HTML
\begin{figure}[!h]
	\centering
	\includegraphics[width=0.25\linewidth]{css}
	\caption{CSS}
	\label{Fig:CSS}
\end{figure}

\subsection{TypeScript}
TypeScript เป็นภาษาสคริปที่มีพัฒนาต่อมาจากภาษา JavaScript ที่เพิ่ม Type System ที่สามารถกำหนดชนิดของตัวแปรได้
เพิ่มความสามารถในการเขียนโปรแกรมเชิงวัตถุ (Object Oriented Programming:OOP) ซึ่งจะทำให้การพัฒนาดเป็นได้ง่ายขึ้น
โดยภาษา TypeScript เป็น transpiler ที่แปลงโค้ดกลับไปเป็น JavaScript ทำให้สามารถทำงานร่วมกับภาษา JavaScriptได้
\begin{figure}[!h]
	\centering
	\includegraphics[width=0.5\linewidth]{ts}
	\caption{TypeScript}
	\label{Fig:TypeScript}
\end{figure}

\subsection{React TypeScript}
React ถูกสร้างขึ้นโดย Facebook เป็น JavaScript library สำหรับสร้าง User interface และมี libraries ที่ช่วยจัดการด้านต่างๆมากมาย ซึ่งทำให้การพัฒนาเว็บไซต์เป็นไปง่ายขึ้น
และเขียนในรูปแบบ SPA (Single Page Application) และเป็น Client Side Rendering
โดยปกติ React จะติดตั้งค่าเริ่มต้นด้วยภาษา JavaScript จึงต้องเปลี่ยนตั้งค่าเป็น TypeScript เพื่อทำให้การพัฒนาง่ายขึ้น
\begin{figure}[!h]
	\centering
	\includegraphics[width=0.5\linewidth]{react}
	\caption{React TypeScript}
	\label{Fig:react}
\end{figure}

\subsection{GraphQL}
GraphQL ถูกสร้างขึ้นโดย Facebook เป็น Query language หรือ ภาษาที่ใช้ในการสืบค้นข้อมูลจาก API เหมือนเป็นตัวกลางที่ใช้ในการจัดการข้อมูลต่างๆ โดยจะ Response กลับมาเป็น JSON สามารถเลือกรับเฉพาะข้อมูลที่ต้องการได้ เพื่อลดปริมาณ Data ในการรับส่งข้อมูล และสามารถเรียกข้อมูลจากหลาย resource จาก request เดียวได้
\begin{figure}[!h]
	\centering
	\includegraphics[width=0.5\linewidth]{graphql}
	\caption{GraphQL}
	\label{Fig:graphql}
\end{figure}

\section{เครื่องมือที่ใช้ในการพัฒนา}
\subsection{Visual Studio Code}
Visual Studio Code: VS Code เป็นโปรแกรม source code editor ในหลายภาษา ที่มีความสามารถและเครื่องมือที่ช่วยเหลือในการพัฒนา เช่น ใช้ code refactoring, ตรวจสอบ syntax, debuging รวมทั้งมี Extension ที่สามารถติดตั้งเพิ่มเติมเพื่อช่วยให้การพัฒนามีความสะดวก รวดเร็วและลดความผิดพลาด
\begin{figure}[!h]
	\centering
	\includegraphics[width=0.20\linewidth]{vscode}
	\caption{Visual Studio Code}
	\label{Fig:vscode}
\end{figure}
\subsection{Git}
Git เป็น version control ที่เป็นระบบที่ใช้จัดเก็บติดตาม และควบคุมการเปลี่ยนแปลงที่เกิดขึ้นกับไฟล์ชนิดใดก็ตาม ช่วยให้การพัฒนางานในทีมเป็นไปอย่างมีระบบ คนในทีมสามารถใช้โค้ดที่เป็นเวอร์ชั่นล่าสุดตลอดเวลา หรือสามารถแก้ไขและแยกสายการพัฒนา (Branch) ออกมาได้
\begin{figure}[!h]
	\centering
	\includegraphics[width=0.4\linewidth]{git}
	\caption{Git}
	\label{Fig:git}
\end{figure}
\subsection{Cloud Build}
Cloud Build คือบริการของ Google ที่ช่วย Build, test และ Deploy อัตโนมัติโดยสามารถตั้งทริกเกอร์ให้ทำงานทุกครังที่ push ขึ้นไปได้
\subsection{Firebase}
Firebase ถูกสร้างขึ้นโดย Google เป็น Platform ที่มีบริการเครื่องมือช่วยเหลือ การพัฒนาแอปพลิเคชัน ในรูปแบบ Serverless โดยมีทั้งแบบใช้ฟรีและ เสียเงินตามที่ใช้ โดยในโปรเจคนี้ได้ใช้บริการเครื่องมือของ Firebase ได้แก่ Cloud Firestore
\begin{figure}[!h]
	\centering
	\includegraphics[width=0.5\linewidth]{firebase}
	\caption{Firebase}
	\label{Fig:firebase}
\end{figure}
\subsubsection{Cloud Firestore}
Cloud Firestore เป็นบริการจัดเก็บข้อมูลโดยโครงสร้างจะเป็นแบบ NoSQL ที่สามารถจัดเก็บข้อมูลในรูปแบบ Document ที่จะผูก Fields กับ Values เข้าด้วยกัน
ซึ่ง Document ก็จะถูกจัดเก็บใน Collections อีกที และสามารถสร้าง SubCollections ใน Document ได้ต่อไปเรื่อยๆ และยังสามารถช่วยเรื่องจัดเรียงข้อมูล (Sorting), การกรองข้อมูล (Filtering), การจำกัดข้อมูล (Limits) และการแบ่งหน้าข้อมูล (Paginate) เป็นต้น
\subsubsection{Firebase Hosting}
Firebase Hosting เป็นบริการ Web Hosting ใช้ในการ deploy เว็บไซต์ขึ้นไปบน Server ให้สามารถเข้าใช้ได้ผ่านทาง Internet รองรับทั้งเว็บไซต์ Static และ Dynamic และมี SSL (Secure Socket Layer) ให้ฟรี

\subsection{Google analytics}
Google analytics ถูกสร้างโดย Google เป็นเครื่องมือสำหรับเก็บสถิตการเข้าเว็บไซต์ และระบุว่าผู้ใช้เข้าเว็บไซต์จากประเทศใด ช่วงเวลาใด และเข้าถึงหน้าใดบ้าง
\begin{figure}[!h]
	\centering
	\includegraphics[width=0.7\linewidth]{google-analytics}
	\caption{Gatsby}
	\label{Fig:gatsby}
\end{figure}

\subsection{Gatsby}
Gatsby เป็น Open source Framework โดยมีพื้นฐานมาจาก React โดยสามารถทำ SSR (Server Side Rendering)
ได้ทำให้การรองรับ SEO และถูก Google search มองเห็นมากขึ้น

\begin{figure}[!h]
	\centering
	\includegraphics[width=0.7\linewidth]{gatsby}
	\caption{Gatsby}
	\label{Fig:gatsby}
\end{figure}

\subsection{Ant design}
Ant Design เป็น Open source UI Framework ในการสร้าง Components สำเร็จรูป ทำให้ไม่เสียเวลาในการสร้างใหม่
ก่อตั้งและพัฒนาภายในประเทศจีน รองรับทั้ง React, Angular และ Vue ที่เป็น Web Framework สำหรับการพัฒนาเว็บไซต์
\begin{figure}[!h]
	\centering
	\includegraphics[width=0.7\linewidth]{ant-design}
	\caption{Ant Design}
	\label{Fig:ant-design}
\end{figure}
\subsection{Apollo}
Apollo เป็น Open source Platform สำหรับพัฒนา APIs ในชั้น communication layer 
โดยใช้ GraphQL Language ในการติดกันระหว่าง 2 ด้าน และมีเครื่องมือช่วยเหลือทั้งด้าน Client (Frontend) และ Server (Backend)

\begin{figure}[!h]
	\centering
	\includegraphics[width=0.5\linewidth]{apollo}
	\caption{Apollo}
	\label{Fig:apollo}
\end{figure}
\subsubsection{Appollo Client (React)}

\subsubsection{Apollo Server}

\subsection{Formik}
เขียนดีไหมวะ


\section{ลักษณะขั้นตอนการทำงาน}
\subsection{Sprint planning}
ทีมจะมีประชุมวางแผนในการจัดสรรงาน (Task) ตาม Sprint backlog ที่ส่วนนี้ทาง System Analyst ได้กำหนดขึ้นมาก่อนแล้ว เมื่อแบ่งงานเสร็จจะมีการประเมินเวลาในการทำงานให้เหมาะสม โดยงานและการประเมินเวลาของแต่ละงาน จะมีการบันทึกลงใน Sprint task board ของ Taiga ซึ่งเป็น Project management tools ที่ทำให้ทุกคนในที่มสามารถติดตามกิจกรรมต่างๆได้ตามรูปที่ \ref{Fig:taiga}
\subsection{Gitflow}
เป็นการใช้ \textbf{Git} ที่เป็นเครื่องมือจัดการ Code collaboration และ Version control โดยแยก Branch เป็น Master Develop และ Feature โดยทีมเลือกใช้บริการ Git ของ \textbf{Bitbucket}

\textbf{Master Branch}\\
เป็น Branch ของ code Production จะเป็นตัวที่ทำการ test และแก้ไขมาเรียบร้อยแล้ว

\textbf{Develop Branch} \\
เป็น Branch ที่ยังอยู่ในการพัฒนา ยังอยู่ในขั้นตอน Test และแก้ไขอยู่ หรือรอเวลาที่จะ Merge รวมเข้ากับ Master Branch หลัง Review เพื่อขึ้น Production

\textbf{Feature Branch}\\
เมื่อจำนวนผู้พัฒนามากขึ้น การพัฒนาใน Develop Branch อาจทำให้ Code ของ มีปัญหากันได้ จึงแยก Branch ไปตาม Feature เช่น feature/daily\_task feature/new\_calendar เมื่อทำเสร็จสิ้น merge รวมเข้ากับ Develop Branch



\subsection{Daily Scrum}
เป็นการบอกความเคลื่อนไหวของงานที่ตนเองได้รับ เพื่อแจ้งความคืบหน้า และแจ้งปัญหาที่ตนเองพบรวมถึงการแจ้งสิ่งทีตนเองได้ทำเสร็จสิ้นไปของเมื่อวาน ให้คนในทีมรับทราบ และช่วยกันแก้ไขปัญหา โดยปกติตามรูปแบบ SCRUM แล้วจะมีการทำ Standup meeting ที่คนในทีมต้องลุกขึ้นยืนและพูดคุยกัน แต่ Daily scrum ของที่บริษัทนี้จะใช้การส่งข้อความลงในแอพลิเคชั่น Slack ดังรูปที่ \ref{Fig:slack}

\subsection{Code review}
เป็นการประเมินการเขียนโปรแกรมเพื่อวิเคราะห์ข้อดี ข้อเสียของโครงสร้างและการทำงานของโปรแกรมที่ได้เขียนไปตลอด Sprint ที่ผ่านมา เพิ่อนำไปปรับปรุงและแก้ไขใน Sprint ถัดไป ในการปฎิบัติจริงกิจกรรมนี้จะทำในช่วง 2-3 สัปดาห์แรกของการปฎิบัติงาน เพิ่อให้มีลักษณะการเขียนโปรแกรมที่สอดคล้องกับคนในทีม หลังจากนั้นจึงมีการประเมินในบางครั้ง

\subsection{Sprint review}
ในช่วงวันสุดท้ายของแต่ละ Sprint จะมีการประชุมเพื่อสรุปสิ่งที่แต่ละคนในทีมได้ทำไป ย้ายงานที่ใช้ระยะเวลานานมากกว่า 1 Sprint ไปไว้ใน Sprint หน้า และพูดถึงปัญหาที่เกิดตลอด Sprint ที่ผ่านมาเพื่อนำไปปรับปรุงในการทำงาน Sprint ถัดไป

\
\begin{figure}[!h]
	\centering
	\includegraphics[width=0.8\linewidth]{taiga}
	\caption{Sprint task board ใน Taiga}
	\label{Fig:taiga}
	\centering
	\includegraphics[width=0.8\linewidth]{slack}
	\caption{Daily meeting ใน Slack}
	\label{Fig:slack}
\end{figure}
\chapter{การออกแบบ และรายละเอียดการพัฒนา}
\label{chapter:experiment}
\section{โครงสร้างและภาพรวมของระบบ}
ระบบจัดการบริหารทรัพยากรบุคคล หรือในชื่อโปรเจค ManyFox พัฒนาขึ้นเพื่อจัดการบริหารภายในองค์กรขนาดเล็ก
โดยระบบที่พัฒนานั้นแบ่งเป็น 6 ส่วน ได้แก่ องค์กรขนาดเล็ก, Slack, เว็บแอปพลิเคชัน (Frontend), ระบบจัดการส่วนหลัง (Backend), Database

\subsection{องค์กรขนาดเล็ก}
องค์กรขนาดเล็กที่มีพนักงาน 2-100คน ที่มีความต้องเก็บข้อมูลพนักงานบน Database มากกว่าในเอกสาร และการจัดการวันลาภายในองค์กรให้มีประสิทธิภาพ
\subsection{Slack}
Slack เป็นแอปพลิเคชันที่ใช้ในการติดต่อสื่อสารกันภายในองค์กร โดยมี Platform ครอบคลุมทั้ง Website, Android, IOS, Windows, Linux และ macOS สมารถสร้าง Channel สำหรับแต่ละหัวข้อได้
และ Slack ยังมีแอปพลิเคชันย่อย ที่ทาง Slack เปิดให้สร้างแอปพลิเคชันย่อยได้ โดยในโปรเจคนี้ จะทำการสร้าง แอปพลิเคชันใน Slack ชื่อ ManyFox

\subsection{เว็บแอปพลิเคชัน (Frontend)}
เว็บแอปพลิเคชัน เป็นส่วนที่เรียกว่า Frontend เป็นส่วนที่ติดต่อกับ User เป็นส่วนที่แสดงข้อมูล การลา ประชุมประจำวัน กิจกรรมต่างๆในองค์กร
หรือดูข้อมูลการลาและจำนวนวันลาที่เหลือของตัวเองได้ และหาก User ที่มีตำแหน่งการบริหาร (ในบริษัท Nextzy technologies คือ C-level Manager และ Senior )
สามารถตอบรับหรือปฎิเสธคำร้องวันลาได้ สามารถเพิ่มลบแก้ไขวันหยุดหรือกิจกรรมในองค์กรได้ สามารถเข้าถึงและแก้ไข ข้อมูลของUserอื่น และสามารถตั้งค่า Workspace ได้ โดยจะเรียกข้อมูลจาก Database จาก API
และนำมาแสดงในหน้า Frontend และเมื่อมี Action ที่ต้องถูกบันทึกลงใน Database ก็จะทำผ่าน API เช่นกัน



\subsection{ระบบจัดการส่วนหลัง (Backend)}
Backend เป็นส่วนจัดการด้านหลัง โดยเชื่อมต่อกับกับ Frontend Slack และ Database โดยแยกเป็น 2 ส่วนดังนี้
\subsubsection{API Web service}
API Web service เป็นส่วนที่เชื่อมต่อกับ Frontend และ Database มีหน้าที่จัดการกับคำร้อง(Request) จาก Frontend
แล้วตรวจสอบประเภทข้อมูล และการอนุญาต ในการอ่าน เพิ่ม ลบ หรือแก้ไขใน Database อาจมีการแปลงรูปแบบข้อมูลก่อนส่งการตอบรับ (Response)
กลับไปหา Frontend

\subsubsection{Slackbot service}
Slackbot service เป็นส่วนที่เชื่อมต่อกับ Slack และ Database  มีหน้าที่จัดการและสร้าง ฟังก์ชั่นที่ถูกประจำในเวลาที่ตั้งไว้ (Cron) ในแต่ละองค์กร สร้างและจัดการ
การตอบสนองอัตโนมัติ (Auto-reply) ในแต่ละองค์กร สร้างและจัดการ หน้าต่าง (Modal) เขียนคำขอร้องลางาน (Off work) หรือ เขียนบันทึกสิ่งต้องทำประจำวัน (Daily meeting)



\section{คุณสมบัติหลักของระบบ}
คุณสมบัติของ ManyFox

\subsection{เข้าสู่ระบบผ่าน Slack}
ผู้ใช้สามารถ เข้าระบบผ่านทางSlackได้ โดยผู้ใช้จำเป็นต้องสร้างหรือมี Workspace ก่อนถึงสามารถใช้งานได้
\subsection{ลงทะเบียน ManyFox}
ผู้ใช้สามรถลงทะเบียน ManyFoxเข้าสู่ Workspace ของ Slack ผ่านทางหน้าเว็บ https://ManyFox.com หรือผ่านทาง Slack โดยกด Add App
ได้โดยผู้ใช้จำเป็นต้องเป็น Administrator ถึงสามารถลงทะเบียนได้

\subsection{เขียนและแก้ไขDaily task}
ผู้ใช้สามารถเขียน ผ่าน Slack โดยพิมพ์คำสั่ง "/task" และมีหน้าต่างแสดงช่องให้พิมพ์สิ่งที่ทำไปแล้ว สิ่งที่กำลังจะทำวันนี้ และ สิ่งที่ต้องทำต่อไปได้และสามารถแก้ไขDaily task ของวันนี้ได้ผ่านการพิมพ์คำสั่ง "/task edit"
\subsection{แสดงผลDaily task}
ผู้ใช้สามารถเรียกดูDaily taskที่ตนเองเขียนได้ เรียงตามวันที่โดยวันล่าสุดขึ้นก่อน หรือสามารถคัดกรองช่วงวันที่ต้องการได้

\subsection{เขียนคำร้องลางานผ่าน Slack}
ผู้ใช้สามรถเขียนคำร้องลางานผ่านทาง Slack โดยพิมพ์คำสั่ง "/offwork" และมีหน้าต่างแสดงให้เลือก ประเภทการลา เหตุผลการลา ระยะเวลาการลา(ทั้งวัน ครึ่งเช้า หรือครึ่งบ่าย) และวันที่ลาได้
และสามารถยกเลิกผ่านทาง Slack โดยการกดปุ่มยกเลิกได้

\subsection{ตอบรับหรือปฎิเสธคำร้องลางาน}
ผู้ใช้ที่มีสิทธิ์ในการตอบรับหรือปฎิเสธคำร้องลางานสามารถกดตอบรับหรือปฎิเสธผ่านหน้าเว็บไซต์ในหน้า Dashboard และหากกดปฎิเสธสามารถกรอกเหตุผลการปฎิเสธได้ เมื่อกดตอบรับหรือปฎิเสธแล้วจะมีแจ้งเตือนถึงผู้เขียนคำร้องได้

\subsection{แสดงผลกราฟการลาภายในบริษัท}
ผู้ใช้ที่มีสิทธิ์ในการตอบรับหรือปฎิเสธคำร้องลางานสามารถเรียกดูกราฟการลาภายในบริษัทภายในปีนี้ได้ สามารถเรียกดูเฉพาะแต่ละเดือนหรือช่วงเดือนได้ และสามารถนำออกมาเป็นไฟล์Excel (.xlsx)ได้

\subsection{จัดการประเภทกิจกรรม}
ผู้ใช้ที่มีสิทธิ์ในการจัดการกิจกรรมภายในองค์กร สามารถเพิ่ม แก้ไขหรือลบประเภทกิจกรรมพร้อมเลือกสีที่จะแสดงในปฎิทิน สามารถเลือกเปิด ปิด ประเภทกิจกรรมที่ต้องการให้แสดงในปฎิทินได้
และสามารถ นำเข้าปฎิทินจาก Pubilc ที่เป็นตัวกลางที่ทาง ManyFox จะจัดการข้อมูลเอง เช่น วันหยุดของประเทศไทย
\subsection{จัดการกิจกรรมภายในบริษัทได้}
ผู้ใช้ที่มีสิทธิในการจัดการกิจกรรมภายในองค์กร สามารถเพิ่ม แก้ไขหรือลบกิจกรรมภายในองค์กร โดยสามารถเขียนชื่อหัวข้อ วันที่ และ ประเภทได้

\subsection{แสดงผลปฎิทินองค์กร}
ผู้ใช้สามารถเรียกดูปฎิทินองค์กร ที่มีข้อมูลวันหยุด-กิจกรรมของบริษัท พนักงานที่ลาในวันนั้น และสามารถเรียกเห็นเฉพาะของตัวเองได้ หรือ เรียกเห็นเฉพาะประเภทการลาได้
\subsection{แสดงผลแจ้งเตือนเหตุการณ์}
ผู้ใช้สามารถเรียกดูการแจ้งเตือนเหตุการณ์ภายในองค์กร เช่น มีบุคคลอนุญาต/ปฎิเสธการลาของตนและเหตุผล และมีการเพิ่มหรือแก้ไขปฎิทินองค์กร เป็นต้น

\subsection{แสดงผลข้อมูลพนักงานและข้อมูลส่วนตัวของตนเอง}
ผู้ใช้สามารถเรียกดูข้อมูลพนักงานของตนเอง (ชื่อ ตำแหน่ง ระดับชั้น และรูปแบบการจ้างงาน) ข้อมูลส่วนตัวของพนักงาน ข้อมูลของจำนวนการลาของผู้ใช้ในปีนี้ และวันลาพักร้อนที่เหลืออยู่
\subsection{แสดงผลข้อมูลพนักงานและข้อมูลส่วนตัวภายในบริษัท}
ผู้ใช้ที่มีสิทธิในการจัดการข้อมูลพนักงาน สามารถเรียกดูข้อมูลพนักงานภายในบริษัท (ชื่อ ตำแหน่ง ระดับชั้น และรูปแบบการจ้างงาน) ข้อมูลส่วนตัวของพนักงาน
ข้อมูลของจำนวนการลาของผู้ใช้ในปีนี้ และวันลาพักร้อนที่เหลืออยู่ และสามารถเรียกดูเฉพาะตามตำแหน่งตามที่ต้องการได้
\subsection{แสดงผลกราฟข้อมูลการลาของพนักงาน}
ผู้ใช้ที่มีสิทธิในการจัดการข้อมูลพนักงาน สามารถเรียกดูกราฟแท่งแสดงข้อมูลการลาของพนักงาน โดยแบ่งตามสีได้ สามารถเลือกดูปีย้อนหลังได้ และสามารถนำออกมาเป็นไฟล์Excel (.xlsx)ได้
\subsection{จัดการข้อมูลพนักงาน}
ผู้ใช้ที่มีสิทธิในการจัดการข้อมูลพนักงาน สามารถแก้ไขข้อมูล ตำแหน่ง ระดับชั้น รูปแบบการจ้างงาน และวันที่เริ่มทำงานของพนักงานภายในบริษัทได้
\subsection{จัดการข้อมูลส่วนตัวได้}
ผู้ใช้สามารถแก้ไขข้อมูลส่วนตัว (เพศ ปีเกิด ที่อยู่ และเบอร์โทร)ของตนเองได้
\subsection{ค้นหาข้อมูลพนักงาน}
ผู้ใช้ที่มีสิทธิในการจัดการข้อมูลพนักงาน สามารถค้นหาข้อมูลพนักงานโดยค้นหาจากชื่อพนักงานได้



\subsection{ตั้งค่าคำตอบสนองแก่คำที่กำหนดไว้อัตโนมัติ}
ผู้ใช้ที่มีสิทธิ์ในการจัดการองค์กร สามารถตั้งค่าคำที่ต้องการให้Bot ตรวจเจอแล้วคำที่ตั้งค่าไว้กลับไป ผ่านทางเว็บไซต์ในหน้า Workspace Setting
\subsection{ตั้งค่าแจ้งเตือนผู้ไม่เขียนDaily task}
ผู้ใช้ที่มีสิทธิ์ในการจัดการองค์กร สามารถตั้งค่าให้แจ้งเตือนแท๊กผู้ที่ไม่เขียนDaily task ในเวลาที่กำหนดผ่านทางเว็บไซต์ Workspace Setting สามารถใส่หรือแก้ไขข้อความต่อท้ายได้ และสามารถเลือก Channel ที่จะแจ้งเตือนได้
\subsection{ตั้งค่าแจ้งเตือนพนักงานที่ลาวันนี้ได้}
ผู้ใช้ที่มีสิทธิ์ในการจัดการองค์กร สามารถตั้งค่าให้แจ้งเตือนพนักงานที่ลาวันนี้ในเวลาที่กำหนดผ่านทางเว็บไซต์ Workspace Setting และสามารถเลือก Channel ที่จะแจ้งเตือนได้
\subsection{ตั้งค่าการลาของพนักงาน}
ผู้ใช้ที่มีสิทธิ์ในการจัดการองค์กร สามารถตั้งค่าประเภทการลาที่ของพนักงานที่สามารถลาได้ตามรูปแบบการจ้างงาน สามารถตั้งค่าประเภทการลาที่ต้องแจ้งก่อนเท่าใด หรือสามารถตั้งค่าจำนวนการของประเภทการลาเท่าใด
\subsection{ตั้งค่ารูปแบบการจ้างานของพนักงาน}
ผู้ใช้ที่มีสิทธิ์ในการจัดการองค์กร สามารถตั้งค่าเพิ่ม แก้ไข หรือลบรูปแบบการจ้างงาน เช่น พนักงานเงินเดือน, นักศึกษาฝึกงาน และ พนักงานอิสระ เป็นต้น
\subsection{ตั้งค่าระดับชั้นของพนักงาน}
ผู้ใช้ที่มีสิทธิ์ในการจัดการองค์กร สามารถตั้งค่าเพิ่ม แก้ไข หรือลบระดับชั้นของพนักงาน และกำหนดสิทธิ์การเข้าถึงของแต่ละระดับชั้นได้
\subsection{ตั้งค่าตำแหน่งของพนักงาน}
ผู้ใช้ที่มีสิทธิ์ในการจัดการองค์กร สามารถตั้งค่าเพิ่ม แก้ไข หรือลบตำแหน่งของพนักงาน เช่น Web Developer, UX/UI Designer, System Analyst และ Business Analyst เป็นต้น
\section{ขั้นตอนการพัฒนา}

การพัฒนาโปรเจคจะแบ่งตาม Sprint โดย 1 sprint จะมีระยะเวลา 2-4 สัปดาห์ ภายในทีมประกอบไปด้วย Project manager, System Analyst, UX/UI Designer, Tester และ Web Developer
โดยการจัดการวางแผน Sprint การเก็บ Requirement และการแบ่งงานเป็นหน้าที่ของ Project manager และ System Analyst โดยเริ่มต้นจาก UX/UI Designer ออกแบบ UI interface ของแต่ละ Page หรือ Feature
เพื่อส่งต่อให้ Web Developer นำไปพัฒนาใน Sprint ต่อไป และเมื่อพัฒนาเสร็จ จะส่งต่อให้ Tester ทดสอบเพื่อจุดบกพร่องใน Sprint ต่อไป
\newpage
\begin{table}[!htbp]
	\centering
	\begin{tabular}{|c|c|c|c|}
		\hline
		\textbf{Sprint}  & Sprint2                   & Sprint 3                  & Sprint 4                  \\ \hline
		\textbf{Feature} & \begin{tabular}[c]{@{}c@{}}Profile\\ Calendar\\ Dashboard\\ Individual Report\\ /task command\\ /offwork command\end{tabular} & \begin{tabular}[c]{@{}c@{}}All profile\\ Edit profile\\ /whooffworks command\end{tabular} & \begin{tabular}[c]{@{}c@{}}Mayfox Install\\ Authentication\\ Notifications\end{tabular} \\ \hline
		\textbf{Sprint}  & Sprint 5                  & Sprint 6                  & Sprint 7                  \\ \hline
		\textbf{Feature} & \begin{tabular}[c]{@{}c@{}}Landing page\\ Feedback\\ Search users\end{tabular} & Leave Setting             & \begin{tabular}[c]{@{}c@{}}All profile\\ Edit profile\\ /whooffworks command\end{tabular} \\ \hline
		\textbf{Sprint}  & Sprint 8                  &                           &                           \\ \hline
		\textbf{Feature} & \begin{tabular}[c]{@{}c@{}}Mayfox Install\\ Authentication\\ Notifications\end{tabular} &                           &                           \\ \hline
	\end{tabular}
	\caption{Sprint plan}
\end{table}


\vskip1em เนื่องจากโปรเจคนี้เริ่มพัฒนาตั้งแต่ช่วงที่นักศึกษาฝึกงานและมีพนักงานร่วมพัฒนา โดยงานที่นักศึกษารับผิดชอบในที่ส่วน Frontend ของ Client ในช่วงฝึกงาน
และรับผิดชอบทั้ง Frontend และ Backend ในช่วงสหกิจ โดยมีการแบ่งหน้าที่รับผิดชอบและงานที่ได้รับมอบหมายดังที่ ปรากฎในตาราง
\begin{table}[!htbp]
	\centering
	\begin{tabular}{cccl}
		\hline
		\multicolumn{1}{|c|}{\textbf{ManyFox System}} & \multicolumn{1}{c|}{\textbf{ผู้รับผิดชอบ Frontend}} & \multicolumn{1}{c|}{\textbf{ผู้รับผิดชอบ Backend}} \\ \hline
		\multicolumn{1}{|c|}{ManyFox Install}         & \multicolumn{1}{c|}{นักศึกษา}                     & \multicolumn{1}{c|}{พนักงาน}                    \\ \hline
		\multicolumn{1}{|c|}{Authentication}          & \multicolumn{1}{c|}{นักศึกษา}                     & \multicolumn{1}{c|}{พนักงาน }                   \\ \hline
		\multicolumn{1}{|c|}{Profile}                 & \multicolumn{1}{c|}{นักศึกษา}                     & \multicolumn{1}{c|}{นักศึกษา \& พนักงาน}          \\ \hline
		\multicolumn{1}{|c|}{Edit profile}            & \multicolumn{1}{c|}{นักศึกษา \& พนักงาน}           & \multicolumn{1}{c|}{นักศึกษา \& พนักงาน}          \\ \hline
		\multicolumn{1}{|c|}{All profile}             & \multicolumn{1}{c|}{นักศึกษา}                     & \multicolumn{1}{c|}{พนักงาน}                    \\ \hline
		\multicolumn{1}{|c|}{Calendar}                & \multicolumn{1}{c|}{นักศึกษา}                     & \multicolumn{1}{c|}{พนักงาน}                    \\ \hline
		\multicolumn{1}{|c|}{Individual Report}       & \multicolumn{1}{c|}{นักศึกษา \& พนักงาน}           & \multicolumn{1}{c|}{พนักงาน}                    \\ \hline
		\multicolumn{1}{|c|}{Daily task}              & \multicolumn{1}{c|}{นักศึกษา}                     & \multicolumn{1}{c|}{นักศึกษา}                    \\ \hline
		\multicolumn{1}{|c|}{Search users}            & \multicolumn{1}{c|}{นักศึกษา \& พนักงาน}           & \multicolumn{1}{c|}{พนักงาน}                    \\ \hline
		\multicolumn{1}{|c|}{Dashboard}               & \multicolumn{1}{c|}{นักศึกษา}                     & \multicolumn{1}{c|}{พนักงาน}                    \\ \hline
		\multicolumn{1}{|c|}{Leave Setting}           & \multicolumn{1}{c|}{พนักงาน}                     & \multicolumn{1}{c|}{พนักงาน}                    \\ \hline
	\end{tabular}
\end{table}

\begin{table}[!htbp]
	\centering
	\begin{tabular}{cccl}
		\hline
		\multicolumn{1}{|c|}{\textbf{ManyFox System}}      & \multicolumn{1}{c|}{\textbf{ผู้รับผิดชอบ Frontend}} & \multicolumn{1}{c|}{\textbf{ผู้รับผิดชอบ Backend}} \\ \hline
		\multicolumn{1}{|c|}{Bot Setting}                  & \multicolumn{1}{c|}{นักศึกษา}                     & \multicolumn{1}{c|}{นักศึกษา}                    \\ \hline
		\multicolumn{1}{|c|}{User Setting}                 & \multicolumn{1}{c|}{พนักงาน}                     & \multicolumn{1}{c|}{พนักงาน}                    \\ \hline
		\multicolumn{1}{|c|}{Feedback}                     & \multicolumn{1}{c|}{นักศึกษา}                     & \multicolumn{1}{c|}{พนักงาน}                    \\ \hline
		\multicolumn{1}{|c|}{Master Calendar Setting}      & \multicolumn{1}{c|}{พนักงาน}                     & \multicolumn{1}{c|}{พนักงาน}                    \\ \hline
		\multicolumn{1}{|c|}{Notifications}                & \multicolumn{1}{c|}{นักศึกษา}                     & \multicolumn{1}{c|}{พนักงาน}                    \\ \hline
		\multicolumn{1}{|c|}{/offwork command}             & \multicolumn{1}{c|}{-}                          & \multicolumn{1}{c|}{พนักงาน}                    \\ \hline
		\multicolumn{1}{|c|}{/whooffworks command}         & \multicolumn{1}{c|}{-}                          & \multicolumn{1}{c|}{พนักงาน}                    \\ \hline
		\multicolumn{1}{|c|}{/task command}                & \multicolumn{1}{c|}{-}                          & \multicolumn{1}{c|}{พนักงาน}                    \\ \hline
		\multicolumn{1}{|c|}{/meeting command}             & \multicolumn{1}{c|}{-}                          & \multicolumn{1}{c|}{พนักงาน}                    \\ \hline
		\multicolumn{1}{|c|}{Auto increase remain offwork} & \multicolumn{1}{c|}{-}                          & \multicolumn{1}{c|}{พนักงาน}                    \\ \hline
		\multicolumn{1}{|c|}{Landing page}                 & \multicolumn{1}{c|}{นักศึกษา}                     & \multicolumn{1}{c|}{-}                         \\ \hline
	\end{tabular}
	\caption{ตารางการมอบหมายงาน}
\end{table}

\subsection{ManyFox Install}
ในการลงทะเบียนติดตั้งใช้งาน Manyfox จำเป็นต้องมี Slack workspace และมิสิทธิ์เป็น Owner หรือ Administrator ใน Slack workspace นั้นถึงสามารถติดตั้งได้
โดยแบ่งออกเป็น 5 ขั้นตอนดังนี้
\subsubsection{introduction}
หน้าแรกสำหรับเข้าสู่การลงทะเบียน เมือกด ADD TO SLACK หากล๊อกอิน Slack ในบราวเซอร์
จะเข้าสู่หน้า Request permission แต่หากไม่ได้ล๊อกอินจะเข้าสู่หน้า Slack login
\begin{figure}[!htbp]
	\centering
	\includegraphics[width=0.8\linewidth]{install-1}
	\caption{introduction}
	\label{Fig:introduction}
\end{figure}
\subsubsection{Slack Login}
Slack login หน้าแรกจะให้กรอก workspace url ของ บริษัท ก่อนเมื่อกด Continue
ระบบจะนำไปตรวจสอบหาว่ามี workspace url นี้จริงหรือไม่ หากมีจะเข้าสู่หน้า Sign in to { \textbf{your Workspace} }
จะมีให้กรอก email และ password ของพนักงานโดย email ของพนักงานจำเป็นต้องอยู่ใน
workspace ของบริษัทเพื่อเข้าสู่ระบบของ Slack และไปในหน้า Request permission ต่อไป
\begin{figure}[!htbp]
	\centering
	\subfigure[Sign in workspace]{
		\includegraphics[width=0.45\linewidth]{install-2}
		\label{Fig:slack-login-1}
	}
	\subfigure[Sign in email]{
		\includegraphics[width=0.45\linewidth]{install-3}
		\label{Fig:slack-login-2}
	}
	\caption{slack login}

\end{figure}
\subsubsection{Request permission}
แอปพลิเคชัน Manyfox จะแสดงคำร้องของการเข้าถึงข้อมูลภายใน Slack
\begin{figure}[!htbp]
	\centering
	\includegraphics[width=0.4\linewidth]{install-4}
	\caption{Request permission}
	\label{Fig:permission}
\end{figure}
\subsubsection{Create initial database}
จะเป็นหน้า Redirect เพื่อยิง API ให้ Backend ดึงข้อมูลจาก Slack มาสร้าง Workspace ภายใน Database
\subsubsection{Install Success}
หากการสร้าง Workspaceเสร็จสิ้น จะขึ้นข้อความ Successfully และเมื่อกดปุ่ม Sign in With Slack จะทำการ Sign in ไปยัง Workspace อัตโนมัติ
\begin{figure}[!h]
	\centering
	\includegraphics[width=0.7\linewidth]{install-5}
	\caption{Install Success}
	\label{Fig:success}
\end{figure}

\subsection{Authentication}
เมื่อสร้าง Workspace ใน Manyfox เสร็จสิ้น สมาชิกทุกใน slack workspace จะสามารถเข้าสู่ระบบภายใต้ workspace ตัวเองได้

\subsection{Profile}
ในส่วน component Profile จะแสดงข้อมูลของ User โดยจะแสดง รูปโปรไฟล์ ชื่อ ตำแหน่ง ระดับชั้น รูปแบบการจ้างงาน สถานะ
จำนวนวันลาของการลาแต่ละประเภทโดยอิงตาม  และแทบสถานะการลาพักร้อน
\vskip1em
โดยด้านขวาบนของ component Profile ยังแบ่งเป็น 2ส่วน ส่วนแรกเป็น สัญลักษณ์ จุด3จุด เมื่อกดจะมี Menu Edit โผล่ออกมาให้เลือก เมื่อกดที่ Edit หน้าข้อมูล Profile จะเปลี่ยนเป็น Edit profile
ส่วนที่2อยู่ด้านล่างของส่วนแรก มีรูปแบบ 2 รูปแบบดังนี้
\vskip1emรูปแบบ \textbf{remain} จะแสดง จำนวนวันลาพักร้อนที่เหลืออยู่

\begin{figure}[!h]
	\centering
	\includegraphics[width=1\linewidth]{profile-remain}
	\caption{Profile รูปแบบ remain}
	\label{Fig:profile-remain}
\end{figure}
\vskip1emรูปแบบ \textbf{more-detail} จะแสดงปุ่ม \textbf{More Detail}" ที่จะลิ๊งไปยัง
/individualreport/\{ \textbf{UserId} \} หน้า Individual report ของ user คนนั้น
\begin{figure}[!h]
	\centering
	\includegraphics[width=1\linewidth]{profile-more}
	\caption{Profile รูปแบบ more-detail}
	\label{Fig:profile-more}
\end{figure}
\\โดยใช้ GraphQL ในการเรียก API จากฝั่ง Backend และชนิดข้อมูลที่ได้ดังนี้\
\subsection{Edit profile}
หลังจากที่เลือก Edit ในหน้า Profile จะเข้าสู่ component Edit profile จะขึ้นรูป โปรไฟล์ User และช่องกรอกข้อความของ ตำแหน่ง ระดับชั้น
รูปแบบการจ้างงาน เพศ วันเกิด ที่อยู่ เบอร์โทร และวันที่เริ่มทำงานของพนักงาน โดยมีข้อมูลเดิมในช่องอยู่แล้ว เมื่อกด Submit จะบันทึกข้อมูลทับของเก่า และ component กลับไปเป็น Profile
\begin{figure}[!htbp]
	\centering
	\includegraphics[width=1\linewidth]{edit-profile}
	\caption{Edit Profile}
	\label{Fig:edit-profile}
\end{figure}
\\โดยใช้ GraphQL ในการเรียก API จากฝั่ง Backend ได้ดังนี้


\subsection{All profile}
ในหน้า All profile จะสามารถเข้าถึงเฉพาะระดับชั้นที่มี permission "see\_report" เท่านั้น
โดยจะแสดง component Profile ในรูปแบบ "more-detail" ของพนักงานทุกคนใน Workspace
โดยสามารถ filter ตามตำแหน่ง และสถานะได้
\begin{figure}[!htbp]
	\centering
	\includegraphics[width=1\linewidth]{all-profile}
	\caption{All profile}
	\label{Fig:all-profile}
\end{figure}
\subsection{Calendar}
Calendar เป็น component ที่ใช้งานทั้งในหน้า Calendar page และ Dashboard page
\vskip1emในหน้า Calendar page จะเป็น Calendar ที่แสดง
event เป็นสีพื้นหลังวันที่ และแสดง ประเภทการลาที่มีในวันนั้นเป็นจุดสี ด้านใต้วันที่
และเมื่อกดที่วันที่จะแสดงข้อมูลวันที่ Event ในวันนั้น และ บุคคลคนที่ลาในวันนั้น พร้อมประเภทการลา และเหตุผล ปรากฎด้านขวามือของ Calendar

\begin{figure}[!htbp]
	\centering
	\includegraphics[width=1\linewidth]{calendar-calendar}
	\caption{Calendar ในหน้า Calendar}
	\label{Fig:calendar-calendar}
\end{figure}
\vskip1emในหน้า Dashboard page จะแสดงเฉพาะ event เป็นพื้นหลังวันที่ และเมื่อกดที่วันที่ จะแสดงกล่อง Calendar Modal ขึ้น
\begin{figure}[!htbp]
	\centering
	\includegraphics[width=0.5\linewidth]{calendar-dashboard}
	\caption{All ในหน้า Dashboard}
	\label{Fig:calendar-dashboard}
\end{figure}
\subsection{Individual Report}
Individual Report page สามารถเข้าถึงเฉพาะระดับชั้นที่มี permission "see\_report" เท่านั้น
โดยจะแสดง component Profile รูปแบบ "remain" และ Individual Report ของพนักงาน โดยดูจาก UserId
ของพนักงานจาก path "/individualreport/\{ \textbf{UserId} \}"
\vskip1em
ส่วน component Individual Report จะแสดง กราฟแท่งจำนวนวันลาในแต่ละประเภทที่ใช้ใน1ปี โดยแบ่งตามเดือน
และแสดง จำนวนลาที่ใช้ใน1ปีทุกประเภท และแต่ละประเภท และสามารถดูย้อนหลังถอยหลัง 2 ปีได้ และปุ่ม Export

\begin{figure}[!htbp]
	\centering
	\includegraphics[width=1\linewidth]{individual-report}
	\caption{All ในหน้า Dashboard}
	\label{Fig:individual-report}
\end{figure}
\vskip1em
ปุ่ม Export เมื่อกดจะดาวโหลดไฟล์ตาราง excel (\.xlsx) ข้อมูลการลาของปีที่เลือกอยู่ โดยจะแสดง วันที่ลา ระยะเวลาการลา ประเภทการลา เหตุผลการลา และการได้อนุมัติ/ปฎิเสธ
\subsection{Daily task}
Daily task เป็น component ที่ใช้ในหน้า Daily task จะแสดงข้อมูลการเขียน daily task ในแต่ละวัน โดยแสดงเฉพาะวันจันทร์-ศุกร์และเรียงจากวันล่าสุดลงไป
\begin{figure}[!htbp]
	\centering
	\includegraphics[width=1\linewidth]{daily-task-1}
	\caption{Daily task}
	\label{Fig:daily-task-1}
\end{figure}
\vskip1em หากมีการลาในวันนั้น จะแสดงข้อมูล เหตุผลการลา ระยะเวลาการลา และประเภทการลา ด้านซ้ายมือของของ กล่อง daily task


\vskip1em หากไม่ได้เขียน daily task ในวันนั้น กล่อง daily task จะเป็นสีชมพูและ มีข้อความ You don\'t write a task ขึ้น

\vskip1em โดยในหน้า Daily task ใช้วิธี Pagination แบบ Cursor Pagination โดยการ Query ครั้งแรกจะได้ข้อมูล 10วันล่าสุดและเมื่อเลื่อนหน้าจอถึงด้านล่างสุดจะ
Query ข้อมูล 10วันถัดมาโดยใช้การส่ง Id จะข้อมูลdaily task ท้ายสุดในการหาข้อมูลถัดมา และสามารถ filter เลือกเฉพาะช่วงวันที่ต้องการได้
\subsection{Search users}
component Search user จะสามารถใช้งานได้เฉพาะผู้มี permission "see\_report" เท่านั้น โดยจะสัญลักษณ์ แว่นขยาย อยู่ด้านขวาบนเมื่อคลิ๊กที่สัญลักษณ์จะมีกล่องค้นหา user สไลด์ออกมาและ เริ่ม Query ข้อมูลของ user ทั้งหมดภายใน workspaceนั้น
เมื่อพิมพ์ข้อความ จะปรากฎรายชื่อ user ประกอบด้วยรูป user ชื่อ และวันลาที่เหลือ โดยค้นหาตามชื่อของ user ที่ประกอบข้อความที่พิมพ์
เมื่อกดที่ user คนใดคนหนึ่ง กล่องจะสไลด์กลับและเปลี่ยนไปหน้า Individual Report ของ user นั้น
\begin{figure}[!htbp]
	\centering
	\includegraphics[width=0.4\linewidth]{search-users}
	\caption{Search users}
	\label{Fig:search-users}
\end{figure}


\subsection{Notifications}
component Notifications เป็นสัญลักษณ์กระดิ่ง และจะมีตัวเลขจำนวนการแจ้งเตือนที่ไม่ได้อ่านอยู่ด้านบนขวาของกระดิ่ง มีตำแหน่งอยู่ที่ด้านบนขวาของหน้าถัดจาก Search user และเมื่อกดคลิ๊กที่กระดิ่ง
จะมีรายการข้อความแจ้งเตือนต่างๆ โดยข้อความที่ยังไม่อ่านจะมีพิ้นหลังสีเทา และข้อความที่อ่านแล้วจะมีพื้นหลังสีขาว
เมื่อกดที่ข้อความนั้นจะเด้งไปหน้าที่เลือกไว้ และเปลี่ยนสถานะเป็นอ่านแล้ว หรือ กดปุ่มด้านบนขวาของรายการ "MARK ALL AS READ" เมื่อกดแล้ว จะเปลี่ยนสถานะข้อความทั้งหมดเป็นอ่านแล้ว

\begin{figure}[!htbp]
	\centering
	\includegraphics[width=0.5\linewidth]{notifications}
	\caption{Notifications}
	\label{Fig:notifications}
\end{figure}

\subsection{Dashboard}
หน้า Dashboard เป็นหน้าสำหรับ ผู้มี permission approve\_request และ add\_calendar ถึงสามารถเข้าใช้งานได้
โดยประกอบไปด้วย 3 component และ 2 Modal ดังนี้
\subsubsection{Request Offworks}
เป็น component แสดงรายการคำร้องขอลางาน จะแสดงข้อมูลการลา รูปโปรไฟล์ ชื่อ ตำแหน่ง ประเภทการลา เหตุผลการลา ระยะเวลาการลา และวันที่ลา ขอพนักงานผู้เขียนคำร้อง
และมีปุ่ม Approve และ X อยู่ด้านขวามือ เมื่อกดปุ่ม Approve หมายถึงยืนยันการลา และส่งแจ้งเตือนไปยังพนักงานผู้เขียนคำร้องทั้งใน เว็บไซต์และ slack
เมื่อกดปุ่ม X จะมีกล่อง Modal ให้กรอกเหตุผลการปฎิเสธการลางาน เมื่อกด submit จะส่งแจ้งเตือนพร้อมเหตุผลไปยังพนักงานผู้เขียนคำร้องทั้งใน เว็บไซต์และ slack
และสามารถคัดกรองร้องผ่านช่วงเวลา (ทั้งหมด ภายในอาทิตย์นี้ ถายในเดือนนี้ และ ภายในปีนี้) หรือคัดกรองผ่านตำแหน่งได้
\begin{figure}[!htbp]
	\centering
	\subfigure[Request Offwork list]{
		\includegraphics[width=0.8\linewidth]{request-offwork}
		\label{Fig:request-offwork-list}
	}
	\subfigure[Reject modal]{
		\label{Fig:reject-modal}
		\includegraphics[width=0.8\linewidth]{reject-modal}
	}
	\caption{Request Offwork}
	\label{Fig:request-offwork}
\end{figure}
\subsubsection{Overall Summary Report}
เป็น component แสดง Area Graph การลางานของพนักงานทั้งหมดภายในปีนี้ แบ่งประเภทการลาตามสี
และสามารถ Export ข้อมูลออกมาผ่านปุ่ม \textbf{Export} ด้านขวาบน เมื่อกดจะดาวโหลดไฟล์ Excel ข้อมูลการลามาในรูปแบบตาราง โดยจะส่งมาตามช่วงที่เลือกไว้
โดยสามารถเลือกรูปแบบได้ 3 รูปแบบดังนี้
\vskip1em รูปแบบ \textbf{Year} จะแสดงเป็นกราฟโดยแบ่งออกเป็น 12 ช่วงตามเดือน
\begin{figure}[!htbp]
	\centering
	\includegraphics[width=0.5\linewidth]{graph-report-1}
	\caption{Overall Summary Report รูปแบบ Year}
	\label{Fig:graph-report-1}
\end{figure}
\vskip1em รูปแบบ \textbf{Month} จะมีเมนูให้เลือกเดือนและแสดงกราฟโดยแบ่งออกตามวันภายในเดือนนั้น
\begin{figure}[!htbp]
	\centering
	\includegraphics[width=0.5\linewidth]{graph-report-2}
	\caption{Overall Summary Report รูปแบบ Month}
	\label{Fig:graph-report-2}
\end{figure}
\vskip1em รูปแบบ \textbf{Custom}  จะมีเมนูเลือกเดือนที่เริ่มต้น จนถึงเดือนที่ต้องการและแสดงกราฟโดยแบ่งตามวันภายในช่วงที่เลือก
\begin{figure}[!htbp]
	\centering
	\includegraphics[width=0.7\linewidth]{graph-report-3}
	\caption{Overall Summary Report รูปแบบ Custom}
	\label{Fig:graph-report-3}
\end{figure}
\subsubsection{Calendar Controller}
เป็น component ไว้สำหรับสร้าง และลบ Event ภายใน Calendar โดยแบ่งเป็นอีก 2 ส่วนคือ Calendar และ Calendar Type
โดยส่วนของ Calendar เมื่อกดที่วันที่จะแสดง Calendar Modal ขึ้นมา
\vskip1em \textbf{Calendar Type} จะแสดงประเภทและสีของ Calendar และสามารถ สร้าง แก้ไขและลบชื่อหรือสีของ Calendar Type ได้
\newpage
\begin{figure}[!htbp]
	\centering
	\includegraphics[width=0.5\linewidth]{calendar-controller}
	\caption{Calendar Controller}
	\label{Fig:calendar-controller}
\end{figure}

\subsubsection{Calendar Modal}
กล่อง Calendar Modal หลังจากกดวันที่ใน Calendar มีไว้เพื่อ แสดงข้อมูล Event ณ วันที่กด หากไม่มี Event ใันวันนั้นจะมีปุ่ม + เมื่อกดจะเปลี่ยนเป็นกล่องสำหรับกรอกข้อมูลสร้าง Event โดยประกอบด้วย
ชื่อหัวข้อ Event  วันที่เริ่ม-วันสุดท้าย Event และ Calendar Type เมื่อกดปุ่ม ADD จะบันทึกข้อมูลลง Database และปิด Modal พร้อม update Query ใน component Calendar
\newpage
\begin{figure}[!htbp]
	\centering
	\subfigure[no Event]{
		\includegraphics[width=0.75\linewidth]{calendar-modal-1}
		\label{Fig:calendar-modal-1}
	}
	\subfigure[Event]{
		\includegraphics[width=0.75\linewidth]{calendar-modal-2}
		\label{Fig:calendar-modal-2}
	}
	\subfigure[ADD Event]{
		\includegraphics[width=0.75\linewidth]{calendar-modal-3}
		\label{Fig:calendar-modal-3}
	}
	\caption{Calendar Modal}includegraphics

\end{figure}
\subsubsection{Calendar Type Modal}
กล่อง Modal สำหรับการเพิ่ม แก้ไข หรือลบ Calendar Type และสามารถกด เปิดหรือปิด Calendar Type นั้นๆ
โดยจะส่งผลให้ข้อมูลCalendar ที่มี Calendar Type ที่ถูกปิดจะถูกซ่อนใน Calendar ไปด้วย
\subsection{Leave Setting}
Leave Setting เป็นหนึ่งในหน้าย่อยของ Workspace Setting เป็นส่วนตั้งค่าเกี่ยวกับการลาของพนักงานโดยจะแบ่งตาม
รูปแบบการจ้างงาน (Employment)โดยสามารถตั้งค่าประเภทการลาได้ดังนี้
\vskip1em \textbf{ลาพักร้อน (Vacation Leave)} สามารถตั้งจำนวนวันลาพักร้อนภายใน1ปี (โดยระบบจะทยอยเพิ่มในวันที่1 ของเดือนถัดไป)
และสามารถตั้งให้สามารถลาได้เฉพาะในอาทิตย์ถัดไป(นับจากวันศุกร์ขึ้นไป ถือว่าเริ่มอาทิตย์ใหม่)
\vskip1em \textbf{ลาป่วย (Sick Leave) }นั้นจะเป็นรูปแบบอนุมัติการลาอัตโนมัติ โดยสามารถตั้งค่าลาป่วยติดต่อกี่วัน ถึงจำเป็นต้องได้รับการอนุมัติ
\vskip1em \textbf{ทำงานที่บ้าน (Work form home)} สามารถตั้งจำนวนวันทำงานที่บ้านมากที่สุดต่อ 1 อาทิตย์
และสามารถตั้งให้สามารถลาได้เฉพาะในอาทิตย์ถัดไป(นับจากวันศุกร์ขึ้นไป ถือว่าเริ่มอาทิตย์ใหม่)
\vskip1em \textbf{ทำงานต่างสาขา (Work from other Branch)} สามารถตั้งให้สามารถลาได้เฉพาะในอาทิตย์ถัดไป(นับจากวันศุกร์ขึ้นไป ถือว่าเริ่มอาทิตย์ใหม่)
\vskip1em \textbf{ลาไปอบรม (seminar Leave)} สามารถตั้งให้สามารถลาได้เฉพาะในอาทิตย์ถัดไป(นับจากวันศุกร์ขึ้นไป ถือว่าเริ่มอาทิตย์ใหม่)

\subsection{Bot Setting}
Bot Setting เป็นหนึ่งในหน้าย่อยของ Workspace Setting เป็นหน้าควบคุมตั้งค่าที่เกี่ยวข้องกับ Slack bot
โดยแบ่งเป็น 3 หน้าดังต่อไปนี้
\subsubsection{Who off works}
หน้าตั้งค่าการแสดงข้อมูลบุคคลที่ลาและข้อมูลการลาภายใน Slack โดยหากเลื่อน Switch ด้านขวาเป็น ON
จะแจ้งเตือนข้อมูลบุคคลที่ลาและข้อมูลการลา ตามเวลาที่กำหนดของทุกวันที่ทำงานและสามารถแก้ไขข้อความด้านใต้ของข้อมูลบุคคลที่ลาและข้อมูลการลาได้
และกำหนดChannel ที่ต้องการให้แสดงผล และสามารถใช้งาน /whooffworks และ /offwork command ภายใน Channel นั้นได้
\subsubsection{Auto-reply}
หน้าตั้งค่าการคำตอบสนองอัตโนมัติจาก Manyfox bot ภายใน Slack โดยสามารถเพิ่ม ลบหรือแก้ไข คำตอบสนองได้มากถึง 50 รูปแบบ โดยแต่ละรูปแบบประกอบไปด้วย
Keyword ที่ต้องการตรวจจับโดยเพิ่มได้มากสุด 5 คำ และ Response คำตอบที่เมื่อจับ1ในคำที่อยู่ Keyword จะแสดงผลคำตอบ (มากสุด 50 ตัวอักษร) โดยหากมีมากกว่า 1 จะสุ่มคำตอบ โดยสามารถเลือก Channel ที่ต้องการสามารถใช้ Auto-reply ได้มากกว่า 1
หรือสามารถเลือก All Channel เพื่อต้องการให้ใช้งานได้ในทุก Channel
\subsubsection{Daily task}
หน้าตั้งค่าการแสดงการเขียน Daily task ภายใน Slack โดยหากเลื่อน Switch ด้านขวาเป็น ON
จะแจ้งเตือนข้อความรายชื่อบุคคลที่ไม่เขียน task ภายในวันนี้ โดยใช้วิธีการแท๊ก ทำให้จะขึ้นแจ้งเตือนแม้ว่าจะปิดการแจ้งเตือนไว้ก็ตาม
โดยจะแสดงตามเวลาที่กำหนดของทุกวันที่ทำงาน และสามารถแก้ไขข้อความด้านใต้ของรายชื่อได้
\vskip1em กล่องขวาจะมีส่วนกำหนด Channel ที่ต้องการให้แสดงผล และสามารถใช้งาน /task command ภายใน Channel นั้นได้
และมีส่วนสำหรับการกำหนด บุคคลที่จะเป็น Whitelist และ Watcher โดย Whitelist หมายถึง บุคคลที่ไม่ได้เขียน task แต่จะไม่ถูกแท็กและมีรายชื่อในแจ้งเตือน (เช่น กรณี ผู้บริหารที่ไม่จำเป็นต้องการโดนแท็ก)
และ Watcher หมายถึงบุคคลที่แม้ว่าจะเขียน task แล้ว ก็ยังถูกแท็กและขึ้นแจ้งเตือน (เช่น กรณี ผู้บริหารที่ต้องการโดนแท็ก เพื่อดูความคืบหน้าพนักงาน)

\begin{figure}[!htbp]
	\centering
	\subfigure[Who off works]{
		\includegraphics[width=0.48\linewidth]{bot-setting-1}
		\label{Fig:bot-setting-1}
	}
	\subfigure[Auto-reply]{
		\includegraphics[width=0.48\linewidth]{bot-setting-3}
		\label{Fig:bot-setting-3}
	}
\end{figure}
\begin{figure}[!htbp]
	\centering
	\subfigure[Daily task]{
		\includegraphics[width=0.75\linewidth]{bot-setting-2}
		\label{Fig:bot-setting-2}
	}
	\caption{Bot Setting}
\end{figure}
\subsection{User Setting}
User Setting เป็นหนึ่งในหน้าย่อยของ Workspace Setting เป็นหน้าควบคุมตั้งค่าที่เกี่ยวกับสถานะของ ตำแหน่ง ระดับชั้น และรูปแบบการจ้างงานโดยแบ่งเป็น 3 หน้าดังต่อไปนี้
\subsubsection{Employment}
Employment หรือ รูปแบบการจ้างงาน เป็นหน้าควบคุมตั้งค่าสำหรับ เพิ่ม ลบ และแก้ไข รูปบบการจ้างงาน
\subsubsection{Level}
Level หรือ ลำดับชั้น เป็นหน้าควบคุมตั้งค่าสำหรับ เพิ่ม ลบ และแก้ไข ลำดับชั้นภายในบริษัท และสามารถตั้งค่าจำกัดการเข้าถึงข้อมูลในแต่ละได้ตามลำดับชั้น
\subsubsection{Job Title}
Job Title หรือ ตำแหน่ง เป็นหน้าควบคุมตั้งค่าสำหรับ เพิ่ม ลบ และแก้ไข ตำแหน่งหน้าที่ภายในบริษัท
\subsection{Master Calendar Setting}
หน้าสำหรับตั้งค่า ของ Manyfox Administrator เท่านั้น สำหรับการเพิ่ม ลบ หรือแก้ไข Public Calendar
\subsection{Feedback}
หน้า Feedback เป็นหน้าสำหรับ แสดงคำถามพบบ่อย (FAQ) และรับฟังความเห็นหรือแจ้งปัญหาของแอพพลิเคชั่น โดยด้านซ้ายจะเป็น FAQ
เมื่อกดที่กล่องหัวข้อคำถาม จะมีข้อความรายะละเอียดโผล่มาด้านล่างของหัวข้อ ส่วนด้านขวาจะเป็นแบบฟอร์มรับฟังความเห็น หรือแจ้งปัญหาโดยต้องกรอก หัวข้อ ข้อความแสดงความคิดเห็น และอีเมลติดต่อกลับ
และสามารถอัพโหลดรูปภาพได้มากถึง 5 รูป โดยแต่ละรูปมีขนาดไม่เกิน 2MB เมื่อกด Submit จะส่งแบบฟอร์มไปเก็บบน Database และมี mail ตอบรับเข้าไปยัง Emailที่กรอกไป

\begin{figure}[!ht]
	\centering
	\includegraphics[width=0.8\linewidth]{feedback}
	\caption{Feedback}
	\label{Fig:feedback}
\end{figure}

\subsection{/offwork command}
เมือเรียกใช้คำสั่ง \textbf{/offwork} บนกล่องแชทใน channel ของ Slack ที่กำหนดในหน้า bot setting ในหัวข้อ Who off works จะปรากฎกล่องแบบฟอร์มสำหรับเขียนคำร้องลางาน โดยมีช่องสำหรับเลือกประเภทการลา ระยะเวลาการลา
เหตุผลการลา วันที่ลา และวันสุดท้ายที่ลา (กรณีลามากกว่า 1 วัน) เมื่อกด submit จะส่งข้อมูลไปเก็บบน Database และมีข้อความรายละเอียดการลาขึ้นในห้องแชท channel ที่เลือก
\begin{figure}[!ht]
	\centering
	\includegraphics[width=0.6\linewidth]{offwork-command}
	\caption{/offwork-command}
	\label{Fig:offwork-command}
\end{figure}
\subsection{/whooffworks command}
เมื่อเรียกใช้คำสั่ง \textbf{/whooffworks} บนกล่องแชทใน channel ของ Slack ที่กำหนดในหน้า bot setting
ในหัวข้อ Who off works จะปรากฎข้อมูลบุคคลที่ลาและข้อมูลการลา ของวันนี้ขึ้นบนห้องแชทใน channel นั้น
\vskip1em และหาก ใช้คำสั่ง \textbf{/whooffworks tmr} จะขึ้นข้อมูลการลาของวันพรุ่งนี้แทน
\begin{figure}[!htbp]
	\centering
	\includegraphics[width=0.6\linewidth]{whooffwork-command}
	\caption{/whooffwork-command}
	\label{Fig:whooffwork-command}
\end{figure}
\subsection{/task command}
เมื่อเรียกใช้คำสั่ง \textbf{/task} บนกล่องแชทใน channel ของ Slack ที่กำหนดในหน้า bot setting ในหัวข้อ Daily task จะปรากฎกล่องแบบฟอร์มสำหรับ การเขียน task
โดยวิธีเขียน task คือเขียน 1 task ต่อ 1 บรรทัด และต้องลงท้ายด้วย ระยเวลาชมที่ใช้ เช่น
\\\textbf{เขียนรายงานสหกิจ (7.5hr)}
และแบ่งช่องในการเขียน daily task เป็น 4 ส่วนดังนี้
\\\textbf{DONE} เขียน task ที่ทำเสร็จของวันที่แล้ว จำเป็นต้องเขียน
\\\textbf{DOING} เขียน task ที่จะทำในวันนี้ จำเป็นต้องเขียน
\\\textbf{TODO} เขียน task สิ่งต้องทำในวันข้างหน้า
\\\textbf{Note} เขียน ข้อความเตือนความจำตนเอง
\vskip1em เมื่อกด Submit จะเก็บข้อมูลบน Database และขึ้นข้อมูล task ที่เขียนไปบนห้องแชท channel ที่เลือก
\vskip1em หากใช้คำสั่ง \textbf{/task me} จะขึ้นข้อมูล task ของผู้ใช้ที่เขียนล่าสุดใน ห้องแชท โดยข้อความนั้นจะมีเพียงผู้ใช้ที่เห็น
\vskip1em หากใช้คำสั่ง \textbf{/task edit} จะขึ้นกล่องแบบฟอร์มในการแก้ไข้ task ล่าสุดที่เขียนไป โดยจะมี task เดิมอยู่บนช่องกรอกข้อมูลไว้แล้ว
เมื่อ submit จะเป็นการบันทึกทับข้อมูล task ของเก่า และขึ้นข้อมูล task ล่าสุดบนห้องแชทอีกรอบ
\begin{figure}[!htbp]
	\centering
	\includegraphics[width=0.4\linewidth]{task-command}
	\caption{/task-command}
	\label{Fig:task-command}
\end{figure}
\subsection{/meeting command}
เมือเรียกใช้คำสั่ง \textbf{/meeting} บทกล่องแชท จะปรากฎกล่องแบบฟอร์มสำหรับการนัดประชุม
โดยต้องกรอกหัวข้อประชุม รายละเอียดการประชุม วันที่ประชุม เวลาเริ่มต้นประชุม ระยะเวลาการประชุม สถานที่จัดประชุม และผู้เข้าร่วมประชุม
เมื่อกด Create จะขึ้นข้อมูลการนัดประชุม และแท๊กและแจ้งเตือนไปยังบุคคลที่นัดประชุมใน slack
\begin{figure}[!htbp]
	\centering
	\includegraphics[width=0.8\linewidth]{meeting-command-1}
	\caption{/meeting-command-1}
	\label{Fig:meeting-command-1}
\end{figure}
\subsection{Auto increase remain offwork}
เมื่อถึงวันที่ 1 ของทุกเดือนระบบจะทำการเพิ่มวันที่จำนวนวันลาพักร้อนที่เหลือ ตามจำนวนที่บริษัทกำหนด และตามประเภทการจ้างงาน
\subsection{Landing page}
หน้าเว็บไซต์สำหรับนำเสนอ Manyfox โดยกล่าวถึงรายละเอียด Feature ต่างๆ
\begin{figure}[!htbp]
	\centering
	

\end{figure}
\begin{figure}[!htbp]
	\centering
	\subfigure[Section-1]{
		\includegraphics[width=0.7\linewidth]{landing-page-1}
		\label{Fig:landing-page-1}
	}
	\subfigure[Section-2]{
		\includegraphics[width=0.7\linewidth]{landing-page-2}
		\label{Fig:landing-page-2}
	}
	\subfigure[Sectio-3]{
		\includegraphics[width=0.7\linewidth]{landing-page-3}
		\label{Fig:landing-page-3}
	}
\end{figure}

\begin{figure}[!htbp]
	\centering
	\subfigure[Section-4]{
		\includegraphics[width=0.5\linewidth]{landing-page-4}
		\label{Fig:landing-page-4}
	}
	\subfigure[Section-5]{
		\includegraphics[width=0.7\linewidth]{landing-page-5}
		\label{Fig:landing-page-5}
	}
	\subfigure[Section-6]{
		\includegraphics[width=0.7\linewidth]{landing-page-6}
		\label{Fig:landing-page-6}
	}
	\caption{Landing page}
\end{figure}
\chapter{ผลการพัฒนา}
\label{chapter:result}

ทีม Manyfox มีการประเมินผลการพัฒนาโดยการวัดผลจาก 2 ด้านคือ เชิงประสิทธิภาพ และ เชิงประสิทธิผล
โดยมีวิธีการในการประเมินผลการพัฒนาดังนี้

\section{การประเมินผลเชิงประสิทธิภาพ}
การประเมินผลเชิงประสิทธิภาพใช้ เว็บไซต์ \textbf{https://web.dev} ที่เป็นเครื่องมือของ google ในการประเมินผลออกมาเป็นตัวเลข และแสดงถึงจุดบกพร่อง 
โดยแยกออกมาเป็น 4 ด้านดังนี้
\begin{figure}[!htbp]
	\centering
	\includegraphics[width=0.8\linewidth]{webdev-logo}
	\caption{Web.dev logo}
	\label{Fig:wed-dev}
\end{figure}
\subsection{Performance}
ประเมินผลจากการตรวจสอบระยะเวลาการโหลดหน้าแรก (ส่วนที่มองเห็นครั้งแรกเมื่อเปิดหน้าเว็บไซต์)
\subsection{Accessibility}
ตรวจสอบปัญหาทั่วไปที่อาจมีผลถึงการะเข้าถึงเนื้อหาบทความ
\subsection{Best practices}
ตรวจสอบรูปแบบโค้ดทั้งหมด ที่ควรเขียนในแบบที่ควรเป็น
\subsection{SEO}
ตรวจสอบ Best practices เฉพาะในส่วน SEO เพื่อให้มั่นใจว่าสามารถค้นหาผ่านทาง Google search เจอ
\newpage
\begin{figure}[!htbp]
	\centering
	\subfigure[ผลการประเมิน Manyfox Web Appication]{
		\includegraphics[width=1\linewidth]{audit-app}
		\label{Fig:audit-app}
	}
	\subfigure[ผลการะประเมิน Manyfox Web Landing page]{
		\includegraphics[width=1\linewidth]{audit-landing-page}
		\label{Fig:audit-landing-page}
	}
	\caption{ผลการประเมิน Manyfox}

\end{figure}

\section{การประเมินผลเชิงประสิทธิผล}
การประเมินผลเชิงประสิทธิผลโดยวัดจากความพึงพอใจของผู้ใช้ เพื่อให้เข้าถึงการรับฟังความเห็นจากผู้ใช้ได้มากขึ้น จึงพัฒนา Feedback page
เพื่อรับฟังความเห็น หรือปัญหาจากผู้ใช้
\chapter{สรุปผลการพัฒนาและข้อเสนอแนะ}
\label{chapter:conclusion}
\section{สรุปผลการพัฒนา}
 โปรเจค Manyfox 

\clearpage
\addcontentsline{toc}{chapter}{บรรณานุกรม}
\bibliographystyle{IEEEtran}
\bibliography{reference}

\startappendix
\chapter{บันทึกการปฎิบัติงานในบริษัท}
\label{chapter:atmosphere}
\section{บรรยากาศของสถานที่}
บริษัท Nextzy Technologies เป็นบริษัทตั้งอยู่ใจกลางเมืองอยู่ใกล้ทั้ง รถไฟ MRT BTS Airport link และเรือคลองแสนแสบ
อยู่ตึกอโศกทาวเวอร์ ชั้น 7 ภายในบริษัทมีห้องประชุมที่เอื้ออำนวยทั้งการประชุม การพรีเซ้นหรือจัดสอนหรือแบ่งปันเรื่องราว 
และเอื้ออำนวยแก่การพักผ่อน และบริษัทยังมีโต๊ะพลูและบอร์ดเกมเพื่อให้พนักงานคลายเครียดและเพิ่มความสัมพันธ์กันภายในบริษัท
\section{บรรยากาศภายในบริษัท}
ภายในบริษัทให้ความยืดหยุ่นแก่การทำงาน เข้างานก่อน 11:00น เลิกงานหลังเวลาที่ตนเข้างาน เพิ่มอีก 9 ชั่วโมงและมีการประชุมบริษัทประจำเดือนเพื่อแจ้งเกี่ยวกับความคืบหน้าภายในบริษัท 
และภายในบริษัทจะเซอร์ไพร์วันเกิดพนักงานจัดโดยพนักงานกันเอง 
\\
\begin{figure}[!htbp]
      \centering
      \subfigure[ประชุมบริษัทประจำเดือน]{
            \label{Fig:meeting}
            \includegraphics[width=0.7\linewidth]{meeting}
      }
     
\end{figure}
\begin{figure}[!htbp]
      \centering
      \subfigure[บรรยากาศตอนนั่งทำงาน]{
            \label{Fig:work}
            \includegraphics[width=0.8\linewidth]{work}
      }
      \subfigure[เซอร์ไพร์เค้กวันเกิดพนักงานด้วยของโปรด]{
            \includegraphics[width=0.7\linewidth]{hbd}
            \label{Fig:hbd}
      }
      \caption{บรรยากาศภายในบริษัท}
\end{figure}
\newpage
\section{กิจกรรมหลังเลิกงาน}
 ภายในบริษัทมีนโยบายกระชับมิตรกันภายในบริษัท เช่น ดูหนังกันทั้งบริษัทโดยทางบริษัทออกค่าใช้จ่ายให้ หรือปาร์ตี้ประจำปี ที่สาขาจากเชียงใหม่จะลงมาร่วมปาร์ตี้ด้วย
\begin{figure}[!htbp]
      \centering
	\subfigure[ไปดูหนังทั้งบริษัท]{
            \includegraphics[width=0.48\linewidth]{movie}
            \label{Fig:movie}
      }
      \subfigure[ปาร์ตี้บริษัทประจำปี]{
            \label{Fig:party}
            \includegraphics[width=0.48\linewidth]{party}
      }
      \subfigure[นัดดูงาน Appleภายในบริษัท]{
            \label{Fig:apple}
            \includegraphics[width=0.48\linewidth]{apple}
      }
      \subfigure[เล่นบอร์ดเกมหลังเลิกงาน]{
            \label{Fig:game}
            \includegraphics[width=0.48\linewidth]{game}
      }
      \caption{กิจกรรมหลังเลิกงาน}
\end{figure}
\section{อบรม Mastering Web Performance Optimization}
Skooldio ได้จัดอบรม  Web Performance Optimization โดย Warat Wongmaneekit 
ที่เป็น Google Developers Expert in Web Technologies (GDE) วันที่ 15 กันยายน พ.ศ.2562 และ ทางบริษัทเห็นว่ามีเนื้อหาอบรม
มีประโยชน์ ต่อหลายโปรเจค จึงส่งพนักงงานไปอบรม 2 คน (นักศึกษาและพนักงาน)
\begin{figure}[!htbp]
	\centering
	\includegraphics[width=0.75\linewidth]{WebOpt3}
	\caption{Poster Mastering Web Performance Optimization}
	\label{Fig:WebPerformance}
\end{figure}
\newpage
\section{เขียนบทความให้ความรู้ด้านเว็บ ใน Blog ของบริษัท }
ทางบริษัทมีนโยบายให้นักศึกษาฝึกงานต้องเขียน เกี่ยวกับงานด้านที่ตนทำใน Blog ของบริษัท อย่างน้อย 1 เรือง
ซึ่งนักศึกษาเองได้เขียนบทความไปถึง 3 เรื่องดังนี้
 \subsection{สร้าง Circle Spinner Menu ด้วย React และ Styled Components}
 \begin{figure}[!htbp]
	\centering
	\includegraphics[width=0.8\linewidth]{blog-1}
	\caption{Circle Spinner Menu}
	\label{Fig:WebPerformance}
\end{figure}
\newpage
 \subsection{มาTerminal Customize ให้เท่ขึ้น 300\% ด้วย Oh My Zsh}
 \begin{figure}[!htbp]
	\centering
	\includegraphics[width=0.8\linewidth]{blog-2}
	\caption{Terminal Customize}
	\label{Fig:WebPerformance}
\end{figure}
 \subsection{Skeleton Screen พริบตาที่ไม่ควรมองข้าม}
 \begin{figure}[!htbp]
	\centering
	\includegraphics[width=0.8\linewidth]{blog-3}
	\caption{Skeleton Screen}
	\label{Fig:WebPerformance}
\end{figure}

\end{document}