\chapter{สรุปผลการพัฒนาและข้อเสนอแนะ}
\label{chapter:conclusion}
\section{สรุปผลการพัฒนา}
 โปรเจค Manyfox เป็น Web Application สำหรับการบริหารจัดการทรัพยากรบุคคลขององค์กร 
 โดยมุ่งเน้นด้านการลาของพนักงงาน และการติดต่อตามงานของพนักงานในแต่ละวัน ซึ่งมีผลลัพธ์ดังนี้
 \begin{itemize}
      \item พนักงานในบริษัทใช้การลาของ Manyfox แทนการลาแบบเก่า
      \item พนักงานในบริษัทเขียน Tasks ในทุกๆเช้าวันที่ทำงาน
      \item พนักงานสามารถรู้วันลาที่เหลือได้โดยไม่ต้องคำนวณเอง หรือถามผ่าย HR
      \item พนักงานทุกคนจะรู้ว่าใครลาวันนี้ในทุกๆเช้า ผ่านการแจ้งเตือนใน Slack
      \item สามารถพัฒนาจนเปิดบริการให้แก้บริษัทลูก และบริษัทอื่นได้
      \item สามารถวิเคราะห์การลาในแต่ละชนิดของพนักงานภายใน 1 ปีผ่านกราฟได้
      \item สามารถวิเคราะห์ช่วงเดือนที่การลามากสุดหรือน้อยสุดในแต่ละชนิดภายใน 1 ปีผ่านกราฟได้
      \item ความสามารถของ Manyfox ตอบโจทย์เฉพาะบริษัทเล็กถึงกลาง และยึดตามวัฒนธรรมของ บริษัท Nextzy technologies มากเกินไป
      \item ระบบมึความยึดติดกับ Slack มากเกินไป ลูกค้าใหม่ต้องใช้เวลานานในการปรับตัว
    \end{itemize}

จากผลลัทธ์ทั้งหมด สรุปได้ว่า Manyfox ตอบโจทย์การทำงานของบริษัท Nextzy technologies 
และ Existing Company ซึ่งเป็นบริษัทลูก ในการลางานของพนักงาน และการติดตามงานในแต่ละวันของพนักงาน
ได้เป็นอย่างดี แต่ยังขาดด้าน Time attendance และ Payroll ซึ่งเป็นส่วนที่บริษัทอื่นให้ความสนใจมากกว่า
\newpage
\section{ข้อเสนอแนะ}
จากการพัฒนาโปรเจค Manyfox มากกว่า 6 เดือนทำให้ทราบปัญหาและข้อจำกัดของทีมและตนเอง จึงมีข้อเสนอแนะดังต่อไปนี้
\begin{itemize}
      \item ควรเก็บ Requirement ที่มีความละเอียดและเก็บในกลุ่มที่หลากหลายมากกว่านี้
      \item ควรทำให้ไม่ยึดติด Slack มากเกินไปหรือ ทำให้ Slack เป็นเพียงตัวเลือกหากไม่มีก็สามารถใช้งานได้
      \item ควรกำหนดขนาด font ที่ใช่งานบ่อย เพื่อวางแผนโครงสร้าง Frontend ระยะยาว
      \item นักศึกษาควรมีพื้นฐานด้าน Backend และ DevOps มากกว่านี้
\end{itemize}